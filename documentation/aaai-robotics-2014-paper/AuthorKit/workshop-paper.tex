%File: formatting-instruction.tex
\documentclass[letterpaper]{article}
\usepackage{aaai}
\usepackage{times}
\usepackage{helvet}
\usepackage{courier}
\frenchspacing
\setlength{\pdfpagewidth}{8.5in}
\setlength{\pdfpageheight}{11in}
\pdfinfo{
/Title (Control of a Robot Arm with Artifical and Biological Neural Networks)
/Author (Abraham Shultz, Holly Yanco, Thomas Shea)}
\setcounter{secnumdepth}{0}  

\newcommand{\superscript}[1]{\ensuremath{^{\textrm{\scriptsize{#1}}}}}
\newcommand{\subscript}[1]{\ensuremath{_{\textrm{\scriptsize{#1}}}}}

 \begin{document}
% The file aaai.sty is the style file for AAAI Press 
% proceedings, working notes, and technical reports.
%
\title{Control of a Robot Arm with Artifical and Biological Neural Networks}
\author{Abraham Shultz, Holly Yanco, Thomas Shea\\
University of Massachusetts Lowell\\
1 University Avenue\\
Lowell, Massachusetts 01835\\
}
\maketitle
\begin{abstract}
\begin{quote}
Put abstract here, play up linkage of biological neuronal network as the inspiration for ANNs, plus use as a controller for robots
\end{quote}
\end{abstract}

\noindent The first line starts with noindent. 

Other sections are indented. 

\section{Introduction}


\section{Methods}

\subsection{Biological Cultured Neuronal Networks}

In order to gather information about about the behavior of neurons, neurobiology researchers grow cultures of neurons outside of the animals that produced them. 
These cultures of neurons allow the researchers to perform experiments that operate directly on the neurons, without the complications that may be caused by the interacting systems of a living organism. 
For example, tetrodotoxin (TTX) prevents sodium channels from acting, which prevents neurons from signaling to each other. 
In culture, the suppression of signaling does not kill the neurons, and can be reversed by removing the TTX. 
However, in organisms, the action of TTX paralyzes the respiratory muscles, which kills the animal.  

Despite the advantages of cultures, they also have drawbacks. 
In culture, a neuronal network does not receive stimulation unless it is provided by the researcher. 
Complete organisms receive stimulation from their senses for their entire lives. 
The incoming sensory information is handled by some degree of pre-existing organization within the brain to form impressions of the outside world. 
By embodying neuronal cultures in a robot, we hope to provide a framework for investigating how incoming signals are integrated by neuronal networks, and how this integration affects the structure of the network. 

\subsection{Experimental Uses}

A Multi-Electrode Array (MEA) is a type of culture dish that provides researchers with a way to monitor the electrical activity of neurons at or near the level of individual cells. 

Because the cells are grown in a culture medium, chemicals can be added to or removed from their environment to modify their ability to signal. 
Research in Thomas Shea's lab \cite{shea2009optimization} has used this method to demonstrate that inhibitory connections are required for learning in cultured neurons. 
For the purposes of this thesis, the neurons under discussion are a culture of disassociated mouse neurons.   

\subsection{Construction}
The MEA itself consists of a glass plate with an array of conductive pads laid out on it, as shown in Figure \ref{fig:MEA_layout}.
Conductive traces extend from each pad to the edges of the plate. 
These conductive pads are used to detect the electrical activity of neurons cultured on the plate.
When a neuron sends a signal, its electrical potential changes, and this change in potential is detected by sensitive amplifiers connected to the traces for pads near that neuron.
The size of each pad is close to the size of a single neuron, so neuron firing can be localized to a single neuron or small group of neurons by determining from which pad the signal came.

To start a culture, the MEA glass is first prepared by coating it with laminin and other proteins that enable neurons to bind to the plate surface. 
The presence of this protein creates a surface that cells are able to stick to, but does not guarantee that a cell will adhere to any specific location and grow there. 
In order to acquire cells, mice must be bred and sacrificed, and the fetal mouse neural tissue must be surgically prepared and chemically treated before being plated on the MEA. 
Fetal mice are used as the cell source, because their neurons are still developing and forming connections. 
The chemical treatment uses enzymes to disassociate the individual neurons. 
The neurons are added as a suspension in liquid medium and given some time to bond, after which the culture medium is replaced, removing any unbonded cells with the old culture medium \cite{wagenaar2006extremely}.
Typical cell suspension densities range from 300 to 2,000 cells per square millimeter, but can reach as high as 80,000 cells per square millimeter \cite{shea2009optimization,ruaro2005toward}.
Extremely sparse cultures tend to have high mortality rates and do not form sufficient connections to display mature signaling patterns \cite{shea2009optimization}.

In addition to culture density, the distribution of cells on the plate can be controlled in other ways. 
One method is to apply the suspension of neurons to the desired regions using a micropipette, resulting in higher neuron density in areas where the drops of suspension were added. 
Another method is to apply the adhesive protein to the plate in a pattern. 
The patterned substrate is created by using a process such as microstamping to apply a pattern of binding proteins, rather than an even coating, to the culture area. 
The cells can only adhere to the areas where the protein is applied, so the resulting plate has areas with high cell density and areas with few or no cells.
The pattern of cells influences the connectivity of the culture \cite{sorkin2006compact}.

When the cells are initially added to the culture, they are not connected. 
For most of the first month in culture, the cells build new connections. 
Starting at around 7 days \emph{in vitro} (DIV) and continuing to around 30 DIV, the connections are not complete, and signaling is dominated by constant, high-amplitude spiking \shortcite{warwick2010controlling}. 
The resulting signals have been described as ``epileptiform."

After the initial period of epileptiform activity, the cells enter a ``mature" phase, characterized by sparse bursts of spikes separated by quiet periods. 
The active bursts may be localized to one region, spread across the culture, or propagate from region to region. 
After 2-3 months of this sort of activity, the culture eventually becomes senescent, and only reacts to stimuli in simple, stereotyped ways \cite{warwick2010controlling}. 
The cells can continue to live for months or even years, assuming that equipment failure or bacterial infection does not kill them \cite{potter2001new}. 

\section{Problem Statement}

Real neurons in culture have a limited life span. 
In the young, epileptiform stage, it is impossible to isolate the neurons' response to stimulus from the constant spike activity, so experiments must be performed after the neuron network is finished developing. 
As a consequence, the window for research on an MEA only lasts as long as the period of mature, complex interaction. 
The period of mature signaling usually lasts 2-3 months, after which the culture enters a senescent stage, where the neuronal responses are minimal or stereotyped. 

The cultures as a whole are difficult to maintain. 
After the culture is plated, the culture medium must be replaced regularly, and the culture as a whole must be maintained at tightly controlled temperature and humidity levels, usually in an automated incubator. 
All of this equipment and maintenance is expensive, and it is all required to prevent the cells dying due to bacterial infection or a hostile environment. 
In addition to the costs of keeping the culture alive, there is a significant investment in hardware to acquire the signal from the MEA, amplify it, and record it. 
All of this equipment is immobile, so physical access to the equipment is required to perform experiments. 

Because the culture is applied to the plate as a suspension of neurons, there are only limited ways to control the eventual location and distribution of neurons. 
The formation of connections between cells in the culture is also stochastic, though some overall organization is emergent from ``rules'' within each cell. 
As a result, each biological culture is unique and cannot be exactly replaced when it dies. 

There are computational methods for determining the approximate wiring of a developed culture, based on the propagation delay of a signal in the culture and the synchrony of activity between different sites in the culture  \cite{erickson2008caged,esposti2008estimation}. 
These methods offer some promise for mapping the connectivity of the dish, but they do not give a complete or fully-accurate map. 
Even if it was possible to completely map the connections of a culture, there is no way to duplicate it, as there is no way to control the growth of individual biological neurons and their axons. 

\subsection{Simulated Cultured Neuronal Networks}

\subsection{Development}

In order to simulate a full MEA, the system must model the dispersal of cells over the surface of the MEA, the networking of those cells, and their activity. 
The first part of the simulation is deciding the distribution of the cells over an area according to the density of the desired culture and the surface area of the MEA plate. 
The process of determining the cell locations is called ``plating."
After the plating simulation has placed the cells, a growth simulation uses the locations of the cells to determine how the individual neurons are connected to form the network. 
In order to decide which neurons are connected, mathematical models based on the observed networking behavior of real neurons are used. 
The plating and growth simulation have been written and several iterations of testing and further development have been performed to bring the output of the simulation into line with the observed behavior of biological neural networks. 
The output of the plating and growth simulations is the connectivity map of a network. 

\section{Neuron models}

For initial development, the cell model used was a simple leaky integrate-and-fire (LIF) model. 
The LIF model was chosen because it is computationally lightweight and displays a sufficient degree of similarity to real neurons to be used in simulation. 
Both Kahng et al. \citeyear{kahng2007stochastic}, and Jolivet, Lewis, and Gerstner \citeyear{jolivet2004generalized} indicate that a LIF model can approximate the spike timing of a living neural network or a more complex mathematical representation of a neuron, to a high degree of accuracy. 
The particular model may have to depend on the cell, but Ostojic, Brunel, and Hakim \citeyear{ostojic2009connectivity} indicates that exponential integrate-and-fire models offer a good match for the behavior of pyramidal cells \emph{in vitro}, so, again, integrate-and-fire models may be both sufficiently accurate and computationally tractable. 
The Jolivet et al. paper also calls out specific portions of the simulated cells' signaling behavior that the LIF model does not accurately capture, so effort in improving the model may be focused in these areas \cite{jolivet2004generalized}. 

However, it should be noted that accuracy of the output of a simulated neuron can be examined for accuracy from the standpoint of timing, or by comparison of the actual electrical output. 
From a timing point of view, a simulated neuron is an accurate representation if the spike output of the neuron matches the conditions and timing that would elicit spikes from a biological neuron. 
The electrical signal has elements, particularly low-amplitude variations, that are not duplicated by LIF models, but can be approximated by more complex models. In this case, not only does the timing of spikes have to be correct, but the voltage output of the neuron must match that of real neurons. 

The LIF model is good for modeling the timing of signals between neurons, but it does not produce biologically plausible action potentials. 
In a biological neuron, the action potential is a sudden spike that appears when the neuron reaches its firing threshold. 
When they reach their firing threshold, LIF neurons simply reset to their rest potential without producing a voltage spike.
The point when they reset is regarded as a spike event by Brian, but is not a spike in the sense of the neuron producing an elevated voltage. 

The result of this lack of spike voltage is that when analyzed by the same tools used to determine the inter-spike intervals in biological networks, the simulation data appears to have relatively few spikes. 
The lack of spikes results in the data used by one of the analysis scripts being very sparse, which in turn appears as unrealistically large values in the output of the analysis scripts, as detailed further in the results section of this thesis. 

The lack of biologically plausible spike voltage in the signal is a problem with any simulation that uses LIF neurons.
LIF neurons are useful, however, because they are not computationally intensive to simulate, which enables scaling to large networks. 
The SIMONE simulator uses LIF neurons, but gets around the lack of spike voltage by adding a biologically plausible voltage spike to the signal when the LIF neuron fires \cite{escola2008simone}.

To obtain more realistic spike voltages, CNS was converted to use an Izhikevich 2-D integrate and fire model instead of basic LIF neurons \cite{izhikevich2003simple}.
The Izhikevich model has parameters that can be configured to reproduce the behavior of many biological neurons. 
The current configuration of the model uses the configuration that Izhikevich calls ``Regular Spiking'' for excitatory neurons and ``Fast Spiking'' for inhibitory neurons, based on the observed behavior of biological neurons (CITE)

\section{Plating simulation}

For the purposes of the plating simulation, the layout of the simulated cells is simplified into a planar grid. 
In a typical MEA, the cells are plated on glass prepared with binding proteins, allowed to bond, and then washed, so any cells that are not in contact with the glass are removed.
As a result, all of the cells in the culture are in a single layer on the glass of the MEA.
Each square of the simulated grid is approximately the size of a single neuron cell body (30$\mu$m), and the full grid is 2500$\mu$m square.
These parameters are configurable in the simulation software, to support different types of cells or configurations of MEA.

Cells are distributed on the grid according to a midpoint displacement fractal algorithm \cite{Fournier1982Stochastic}. 
Kahng et al. \citeyear{kahng2007stochastic} does not provide details on their model of cell distribution beyond a description of the plating process.
For CNS, a midpoint displacement fractal was chosen to set the distribution of cell adhesion probabilities because of the similarity of its results to turbulent flows. 
The uneven distribution of cells in dishes is supported by the uneven areal density of cultures as seen in Shea \citeyear{shea2009optimization}. 
The plasma fractal provides a real-valued probability of each point on the dish being occupied by a cell. 

As an alternative to the plasma fractal, CNS also allows the use of an image to specify the cell occupancy probabilities. 
The red channel of the image is mapped to the grid of points on the dish, with the saturation of color at each point used as the probability of that point containing a cell. 
The image can contain stripes or other patterns, which can be used to simulate micropatterning of the adhesive protein substrate of the dish, or micropipetting of the suspended neurons to specific locations in the dish. 

After the probabilities are determined, using either an image or a plasma fractal, the plating simulation then marks each location as occupied or not, based on the probabilities of a location having a cell and the density of cells in the plating solution. 
Those locations that are marked as occupied are treated as having a cell on them. The others are assumed to be empty space. 

After the cell locations are determined, there are a series of pruning steps that are intended to simulate cell deaths in the culture. 
Not all of the cells from the initial plating survive to maturity. 
In biological cultures, 45-60\% of the cells die before the network is done wiring itself, within approximately the first 17 DIV \cite{erickson2008caged}.
Because so many of the cells die off, they do not need to be considered when the simulation begins to determine network connectivity. 
In order to model the early cell mortality, the locations that the simulator has marked as occupied are decimated based on the observed survival probabilities of cells in culture. 
The survival rate, expressed as a percentage, is a configurable parameter of the simulator. 
At present, the pruning function assumes that all cells are equally likely to die, but this function could be updated to bias cell survival rates in a number of ways, such as making cells that are near other cells more likely to survive. 
Obviously, such biases should be supported by observation of biological neuronal networks. 


\section{Connectivity and Growth}

Axons may be of any length. 
In humans, the sciatic axon reaches from the base of the spine to the big toe, nearly a meter. 
Because the active area of an MEA is around 2mm\superscript{2}, it is possible for any cell to be connected to any other cell, resulting in $N\times(N-1)$ possible connections among $N$ cells. 
Typical cell suspension densities are in the range of 300-2000 cells/mm\superscript{2}, resulting in 1200-8000 cells in the active area of the MEA and so millions of possible connections \cite{wagenaar2006extremely}.
However, there are limits on cell growth and networking which make the computation of the network connectivity more tractable. 
Chemical interactions between cells restrict the number of connections that should be considered when developing the connectivity of the dish. 

Kahng, Nam, and Lee \citeyear{kahng2007stochastic} provides a model based on observation of chemotaxis in developing neurons, but simplified into a stochastic model. 
The growing end of an axon moves in a random walk on a grid. 
Each step may take it in any of 8 directions: up, down, left, right, or the four diagonals.
If, after making a step, the walking point is with 20$\mu$m of a dendrite of another neuron, the two neurons are considered connected. 
The paper indicates that the probability of a connection between two cells is effectively a function of the distance between them which makes it unlikely a cell will connect to itself, but likely it will connect to neighbors, and unlikely that it will reach very far \cite{Segev2000185}. 

Gafarov \citeyear{gafarov2006self} suggests that the spiking activity of developing neurons also causes the release of the chemicals that will attract developing axons towards a cell. 
The effect of these chemicals on the developing axons is interesting because it proposes a function for epileptiform activity. 
The high frequency signaling in early development would cause higher release of attractive chemicals, and so drive the formation of early connectivity. 
Stimulation also increases activity of the stimulated neurons, and so the development of connectivity in the growing culture can be guided by stimulation. 
Specifically, a specific spatial pattern of stimulation will encourage growth around the neurons activated by the stimulus, thus driving an increase in connectivity to those neurons \cite{zemianek2012accelerated}. 

Our simulator uses a Gaussian distribution to model the probability of a pair of cells connecting based on the straight-line distance between them, with parameters set to maximize connectivity around 200$\mu$m from the cell body. 
The Gaussian distribution also provides a limitation on the number of possible connections that must be considered by the program during the growth simulation. 
If the distance between two cells is so large that the probability of a connection between them is vanishingly small, it may be disregarded when the network is being laid out, thus saving computation time. 

In addition to the limits imposed by chemotaxis, observations of the connections in MEAs performed by confocal microscopy indicate that only 20-50\% of the possible connections are made. 
Once some connectivity threshold has been reached, a neuron will not connect to any other neurons. 
The restriction on connectivity sets an upper limit on the out-degree of neurons, that is, the number of other neurons that they form connections to, considered from the perspective of the connecting neuron. 
A number of possible limitations on the out-degree of neurons have been found in the literature.  

Segev et al. \citeyear{segev2003formation} indicates that cultured neurons tend to send out approximately 10 neurites to connect to other cells. 
The number of outgoing connections provides a good upper bound on the number of connections made by simulated cells, but it seems unrealistic to assume that simply stopping at 10 connections will build a biologically plausible network.

Patel, Scott, and Meaney \citeyear{patel2012dynamic} indicate that the out-degree of neurons can be modeled using a Poisson distribution with a mean of 22. 
Allowing such a stochastic distribution of connectivity will cause some neurons to be extremely well connected while others are less connected. 
Such patterns of connectivity are seen in biological cultures, so the Poisson distribution is used in the simulation to set the out-degree of the neurons. 
Once a neuron forms a number of connections equal to its selected out-degree, no further connections from that neuron are considered, although other neurons may still form connections to it. 

Once the model is completed, it may be run, and voltage and spike train data collected from it. 
Since each neuron is in a known location in the simulated culture, the simulation selects the neurons located on or near the conductive pads for a given MEA layout, and records data from those neurons. 
The Brian simulation software also supports logging of potentials of arbitrary individual neurons as well as collection of spike data from any neuron or set of neurons, including the the possibility of logging the membrane voltage of every neuron in the entire simulated population.
Biological cultures do not support this level of logging detail.  

\subsection{Computer Interface}

In order to read neuronal signals from the culture, the MEA is placed in a MEA1060-INV amplifier manufactured by Multi Channel Systems GmbH. 
The amplifer has 60 channels, one for each contact pad of the MEA.

The signals from the amplifier are acquired by a PCI-6071E DAQ card manufactured by National Instruments. 
The card is controlled through the Linux COntrol and MEasurement Device Interface (COMEDI), which provides an open source library for collecting data from DAQ cards. 

The acquisition and processing software is maintained as a collection of ROS nodes.
The node which acquires data from the DAQ card is called ``Zanni". 
Zanni simply samples the card 1000 times per second and outputs the current value in volts of each channel of the MEA. 
A collection of 60 voltage values is referred to as a ``dish state'' because it represents the electrical activity of the MEA at a specific instant in time. 

Dish states are collected by a ROS node which uses the time series of voltages in each channel to determine the mean and standard deviation of the electrical signal. 
Any time that the signal increases beyond 3 standard deviations from the mean, the channel is considered to be ``spiking''. 
A spike on a channel indicates that a neuron near that channel has produced an action potential. 

For each channel in the dish, if a spike is detected on that channel, the activation at that site is incremented. 
Activation decays exponentially over time 

TODO activation increment and decay functions here

The activation over the entire culture is normalized to the range 0.0 - 1.0 by applying

TODO normalization function here

The resulting vector of 60 values is the activation vector for the dish at a specific time.
Every 0.2 seconds, the activation vector of the culture is compared to a pair of pre-selected activation vectors. 
The pre-selected vectors are a ``right'' and ``left'' vector, with the ``left'' vector having maximum activation at all pads on the left side of the dish, and zero elsewhere, while the ``right'' vector has maximum activation at all pads on the right side of the dish and zero elsewhere. 

\begin{table}
\begin{tabular}{cccccccc}
 & 1.0 & 1.0 & 0.0 & 0.0 & 0.0 & 0.0 &  \\ 
1.0 & 1.0 & 1.0 & 0.0 & 0.0 & 0.0 & 0.0 & 0.0 \\ 
1.0 & 1.0 & 1.0 & 0.0 & 0.0 & 0.0 & 0.0 & 0.0 \\ 
1.0 & 1.0 & 1.0 & 0.0 & 0.0 & 0.0 & 0.0 & 0.0 \\ 
1.0 & 1.0 & 1.0 & 0.0 & 0.0 & 0.0 & 0.0 & 0.0 \\ 
1.0 & 1.0 & 1.0 & 0.0 & 0.0 & 0.0 & 0.0 & 0.0 \\ 
1.0 & 1.0 & 1.0 & 0.0 & 0.0 & 0.0 & 0.0 & 0.0 \\ 
 & 1.0 & 1.0 & 0.0 & 0.0 & 0.0 & 0.0 &  \\ 
\end{tabular}
\caption{Example of an activation vector for activity on the left side of the dish, with each vector element in its location on the dish} 
\end{table}

The comparison is a simple calculation of euclidian distance between the left and right vectors and the current activity vector of the culture. 
If the distance from the current activation vector to the left vector is less than the distance to the right vector the arm will be commanded to move left. 
Similarly, if the current vector is closer to the right than the left vector, the arm will be commanded to move right. 
In either case, the difference between the distances must be large enough to overcome a dead band, or the arm is not instructed to move at all. 



\section{Results}

Control with biological cultures

Noise and noise suppression

Control with an artificial culture

\section{Discussion}

\bibliography{workshop-paper}
\end{document}
